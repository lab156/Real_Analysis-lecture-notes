\noindent\textbf{Real Analysis II Assignment 8 \hspace{\fill} Luis Berlioz}
\problem{58}
\subpart{(c)}
First we prove that $\RRR(T)^\perp = \NNN(T^*)$. Let $y\in \RRR(T)^\perp$; this means that for all $x\in \HH$ we have:
$$\inprod {Tx}y=0 = \inprod x{T^*y}$$
Since $x$ can take any value, let it be $x=T^*y$ then:
$$\inprod {TT^*y}y=0 = \inprod {T^*y}{T^*y}=\v T^*y\v^2$$
This means that $\RRR(T)^\perp \subset \NNN(T^*)$.

On the other hand, if $y\in \NN((T^*)$ then $T^*y=0$, and this means that $\forall x\in \HH$:
$$0=\inprod x{T^*y} = \inprod {Tx}y $$
Thus $y\in \RRR(T)^\perp$ and this enables us to conclude that $\RRR(T)^\perp = \NNN(T^*)$.

Secondly, we show that $\overline{\RRR(T^*)}^\perp = \NNN(T)$. Since  $T^{**}=T$, and $T^*$ is just another linear operator we can say:
$$\RRR(T^*)^\perp = \NNN(T)$$
Taking perpendicular complements on both sides and using that $(E^\perp)^\perp$ is the smallest closed subspace that contains $E$
$$\overline{\RRR(T^*)}= \NNN(T)^\perp $$
Note that the closure is necessary, as the range of even continuous linear operators may fail to be closed; to wit: take $T\from \ell_\infty \to \infty$ and $x\in \ell_\infty$ defined as:
$$T(x) = \left( \frac{x_j}{j} \right)_1^\infty$$
Then $\v T \v=1$ and the image is not closed\footnote{This example is from: http://math.stackexchange.com/questions/305503/what-are-the-range-and-the-norm-of-this-bounded-linear-operator}

\subpart{(d)}
Assuming $T$ is invertible, first let us suppose that $T^*=T^{-1}$. Then $\forall x,y\in \HH$ we have the following:
$$\inprod{Tx}{Ty} = \inprod x{T^*Ty} = \inprod xy$$
Thus, $T$ is unitary.

Now let us suppose that $T$ is a unitary map (we are still assuming it is invertible). 
We will show that $\v y - T^*Ty\v=0$ 
\begin{gather*}
\v y - T^*Ty\v^2=\inprod{y- T^*Ty}{y- T^*Ty}\\
=\inprod yy - \inprod y{T^*Ty} - \inprod {T^*Ty}y + \inprod{T^*Ty}{T^*Ty}
\end{gather*}
Which is zero since a unitary map preserves the inner product. 
Since exactly the same procedure can be done to prove that $\v TT^*y- y\v=0$; we conclude that $T^*=T^{-1}$.

\problem{62}
\subpart{(a)}
We will follow the text's hint of modifying the proof of Theorem 2.26. 
Our first step will be to show that $L^2[0,1] \subset L^1[0,1]$. If $f$ is any function in $L^2[0,1]$ we define:
$$\bar f = \max\{ |f|,1\} \geq |f|$$
We need to show that $\int |f| <\infty$, if $E$ is the set where $|f|$ is less than 1:
$$\int |f| \leq \int \bar f = \int \chi_E + \int_{[0,1]\backslash E} |f| \leq \mu(E) + \int_{[0,1]\backslash E} |f|^2 <\infty$$

Next we show that the same sequence $\{\phi_n\}$ used in the proof of Theorem 2.26 (which is the same used in Theorem 2.10) also works in this case. 
We will use the fact that the $\{\phi_n\}$, being simple functions,  are all $L^2[0,1]$ integrable. 
We will use the dominated convergence theorem with the sequence $|f-\phi_n|^2(x) \to 0$ and the bound:
$$|f-\phi_n|^2\leq 4|f|^2$$
Implying that:
$$\int |f-\phi_n|^2 \to 0$$
Thus, the convergence in the $L^1[0,1]$ metric implies convergence in $L^2[0,1]$. Therefore the $\phi_n$ are a sequence in $L^2[0,1]$ that converges to any $f\in L^2[0,1]$. Finally, as in the proof of Theorem 2.26, the continuous functions $C[0,1]$ can approximate all $\phi_n$. 

\subpart{(b)}
This follows from the result seen in class (Stone-Weistrass approximation theorem). The argument goes as follows: let $f\in L^2[0,1]$; there is a sequence $\{\phi_n\}$ of simple functions that converge to $f$. Additionally, there is a sequence in $C[0,1]$ converging to each $\phi_n$. Finally, there are sequences of polynomials converging to each of the objects above, and this set has $f$ as a limit point.

\problem{63}
\subpart{(a)}
Let $x\in \HH$ be fixed but arbitrary, and say $\{u_n\}$ is an orthonormal set. By Bessel's inequality:
$$\sum_{j=1}^\infty |\inprod x{u_n}|^2 < \infty$$
This means that $\inprod x{u_n}\to 0$. Notice that the functionals $\inprod x\cdot$ are all elements in the dual.
\subpart{(b)}
Let $\{v_n\}$ be an orthonormal basis of $\HH$. Then $\{v_n\}\subset S$ and any $x\in B$ can be approximated with a sum of the form:
$$\sum_{j=1}^n \inprod x{v_n}v_n \to x$$
Any neighborhood of $x$ in the weak topology is represented with a list of vectors $B_{y_1,\ldots y_k}$. Now, taking the $y_i$ of unitary lengths gives that result.
