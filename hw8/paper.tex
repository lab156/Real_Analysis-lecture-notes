\problem{58}
\subpart{(c)}
First we prove that $\RRR(T)^\perp = \NNN(T^*)$. Let $y\in \RRR(T)^\perp$; this means that for all $x\in \HH$ we have:
$$\inprod {Tx}y=0 = \inprod x{T^*y}$$
Since $x$ can take any value, let it be $x=T^*y$ then:
$$\inprod {TT^*y}y=0 = \inprod {T^*y}{T^*y}=\v T^*y\v^2$$
This means that $\RRR(T)^\perp \subset \NNN(T^*)$.

On the other hand, if $y\in \NN((T^*)$ then $T^*y=0$, and this means that $\forall x\in \HH$:
$$0=\inprod x{T^*y} = \inprod {Tx}y $$
Thus $y\in \RRR(T)^\perp$ and this enables us to conclude that $\RRR(T)^\perp = \NNN(T^*)$.

Secondly, we show that $\overline{\RRR(T^*)}^\perp = \NNN(T)$. Since  $T^{**}=T$, and $T^*$ is just another linear operator we can say:
$$\RRR(T^*)^\perp = \NNN(T)$$
Taking perpendicular complements on both sides and using that $(E^\perp)^\perp$ is the smallest closed subspace that contains $E$
$$\overline{\RRR(T^*)}= \NNN(T)^\perp $$
Note that the closure is necessary, as the range of even continuous linear operators may fail to be closed; to wit: take $T\from \ell_\infty \to \infty$ and $x\in \ell_\infty$ defined as:
$$T(x) = \left( \frac{x_j}{j} \right)$$
Then $\v T \v=1$ and the image is not closed\footnote{This example is from: http://math.stackexchange.com/questions/305503/what-are-the-range-and-the-norm-of-this-bounded-linear-operator}

\subpart{(d)}

