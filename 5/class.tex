Remember that in $\RR^n$ to define a norm is equivalent to define a unit ball.

\subsection{Norm on product of Vector Spaces}
\begin{examples}
The norm on the space $\vesp\times \vespy$ can be defined by:
\begin{align*}
\v (x,y) \v &= \max\{\v x\v + \v y \v\}\\
            &= \left(\v x\v^2 + \v y \v^2\right)^(1/2)\\
            &= \v x \v + \v y \v
\end{align*}
\end{examples}

Let $T\from \vesp \to \vespy$ be a linear operator; then $\v T\v \leq \lambda \iff TB_\vesp\subset \lambda B_\vespy=B(0,lambda)$.

\begin{remarks}
Remember that Continuous linear is equivalent to bounded.

Let $T\from \vesp \to \vespy$ be a bijection; then $T^{-1}$ is continuous when there exists a constant $c>0$ (named for the small $c$ in the definition of equivalent norms) such that for all $x\in vesp$,
$$c\v x\v \leq \v Tx\v$$
This is the opposite order that is used when proving that bounded linear implies continuous.
\end{remarks}

\begin{ddef}[Quotient of Spaces]
$\vesp$ normed spaces $\MM\subset \vesp$ is a subspace.
$$\vesp/\MM= \left\{ x+ \MM \from \forall x\in \vesp\right\}$$
\end{ddef}
\begin{remarks}
Two way to ``think'' about the quotient spaces:
\begin{itemize}
\item The set of affine spaces; that is, the elements are all the parallel spaces to $\MM$.
\item Also equivalent to the intersection with the complementary (orthogonal) subspace.
\end{itemize}
\end{remarks}

\begin{ddef}[Norm of the Quotient Space]
The norm of $y \in \vesp/\MM$ is:
$$\v y\v= \inf_{m\in \MM} \v x+m\v$$
\end{ddef}

\begin{corol}
\begin{enumerate}
\item The canonical projection $\pi\from \vesp \to \vespy$ has norm $\v x\v =1$.\item $U\subset \vesp/\MM$ is open iff $\pi^{-1}(U)$ is open.
\end{enumerate}
\end{corol}

\begin{teorema}
$\vesp$ is complete iff $\vesp/\MM$ is complete.
\begin{proof}
Use the absolute convergence and Cauchy criterion.
\end{proof}
\end{teorema}
