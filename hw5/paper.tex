\problem{34}
\subsection*{Part (a)}
Let $x_n \to 0$ as a sequence in the space $L_k^1[0,1]$.
Then $\imath(x_n)=x_n$ also converges to $0$ in $C^{k-1}[0,1]$ simply because its norm also separates points.

\subsection*{Part (b)}
\problem{38}
First we check that $T$ is a linear; let $\alpha,\beta$ be scalars, and $x,y\in \vesp$ then it checks:
$$\lim_{n\to \infty} T_n(\alpha x+\beta y) = T(\alpha x+\beta y)$$
Since each $T_n$ is linear and convergent we can conclude:
$$T(\alpha x+\beta y)=\alpha T_nx +\beta T_n y \to \alpha Tx + \beta Ty$$
Next to check that $T$ is bounded, we use tha Uniform Boundedness Principle.
Note that $\{T_n \mid n\in\NN \} \subset L(\vesp, \vespy)$. 
This implies that for any fixed  $x\in \vesp$, the set composed of $\{\v T_n x\v from n\in \NN\}$ is bounded.
Finally this means that $\v T\v$ which has to be less than $\sup_{n}\v T_n\v$ is finite.

\problem{41}
Let $\BB=\{u_k\from k\in \NN\}$ be a basis of the vector space $\vesp$.
Any finite subset of $\BB$ (namely $U$) spans a subspace that is:
\begin{enumerate}
    \item Finite dimensional 
    \item Closed 
    \item No-where dense. 
\end{enumerate}
To check the last one, note that any open ball around the origin contains some multiple of some $u_k \notin U$.
This means that the span of $U$ has empty interior.

By definition the elements of $\vesp$ are always finite linear combination; so:
$$\vesp = \cup_{U\subset \BB} \spann U$$ 
Where $U$ are \emph{U} subsets as above. 

Let $\vespy$ be the completion of $\vesp$. 
By the Baire Category Theorem, $\vespy\neq \vesp$.

\problem{Extra}
We will show that the two given bases generate the same topology.
First let $B(x,\delta)$ be any ball centered at an arbitrary $x$ and for any $\delta>0$. 
Let $N$ be such that:
$$\sum_{k=N+1}^\infty 2^{-k} < \delta/2$$
Then:
$$\sum_{k=N+1}^\infty 2^{-k} \frac{d(x_k,y_k)}{1+d(x_k,y_k)}< \delta/2$$
Need to show that $U_{\delta/N,N}\subset B(x,\delta)$.
Take $y=(y_k) \in U_{\delta/N,N}$ then for $1\leq k \leq N$ we have $(1+d(x_k,y_k))^{-1}d(x_k,y_k)\leq d(x_k,y_k) < \delta/N$ thus:
\begin{align*}
\rho(x,y) &= \sum_{k=1}^N 2^{-k} \frac{d(x_k,y_k)}{1+d(x_k,y_k)} + \sum_{k=N+1}^\infty 2^{-k} \frac{d(x_k,y_k)}{1+d(x_k,y_k)} \\
          &\leq \frac 12 \delta + \delta/2 = \delta
\end{align*}

Secondly, let $U_{\epsilon, n}$ be any element of the base of a point $x=(x_k)$. We will show that:
$$B(x,\frac{2^{-n}\epsilon}{1+\epsilon}) \subset U_{\epsilon, n}$$
To do this, observe that if $y$ is such that:
$$\rho(x,y)<\frac{2^{-n}\epsilon}{1+\epsilon}$$
Then for $1\leq k \leq n$:
$$2^{-k} \frac{d(x_k,y_k)}{1+d(x_k,y_k)} < \frac{2^{-n}\epsilon}{1+\epsilon}$$
Note that $2^{-n}\leq 2^{-k}$ so:
$$ \frac{d(x_k,y_k)}{1+d(x_k,y_k)} < \frac{\epsilon}{1+\epsilon}$$
And since for any positive number the map $d\mapsto (1+d)^{-1}d$ is one-to-one:
$$d(x_k,y_k)<\epsilon$$
$\therefore \ y\in U_{n,\epsilon}$.
