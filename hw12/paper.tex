\noindent\textbf{Real Analysis II Assignment 12 \hspace{\fill} Luis Berlioz}
\problem{6.2}
\subpart{(c)}
If we assume that $\v f_n - f\v_\infty\to 0$, then for all $\epsilon>0$ there exists $N\in \NN $ such that for all $n\geq N$:
$$\v f_n - f\v_\infty < \epsilon$$
This is the same as saying:
$$\inf_{a\geq 0}\left \{ a\from \mu\{ x\from | f_n(x) - f(x)| > a\}=0\right\}<\epsilon$$
In particular this means that for $n\geq N$:
$$\mu\{ x\from |f_n(x) - f(x)|\geq  \epsilon \}=0$$
Even though the measure of the set is 0 for all $n\geq N$, the sets might change completely as we change $n$ so we have to take:
$$H(\epsilon) =\bigcup_N^\infty  \{ x\from |f_n(x) - f(x)|\geq  \epsilon \}$$
Observe that $H(\epsilon)$ is still of measure 0. 
Moreover, take $E^C = \cup_{k=1}^\infty H(1/k)$ by the same reason as before, $\mu(E^C)=0$.

Next we argue that $f_n\convu f$ on $E$. 
For any $\epsilon >0$ take $\epsilon > 1/k$; by the argument above, there is an $N\in \NN$ such that for $n\geq N$:
$$H(1/k)= \bigcup_N^\infty  \{ x\from |f_n(x) - f(x)|\geq  \epsilon \}\subset E^C$$
Therefore, $\forall x\in E$ it must be the case that $|f_n(x) -f(x)| < \epsilon$.

Conversely, using the same setting as in the textbook and in the hints:
$$\v f_n -f\v_\infty \leq \v (f_n -f) \chi_E\v_u\to 0$$ 

\subpart{(d)}
If $f\in L^\infty$ then $\v f\v_\infty <\infty$ and this means that $f$ is in the same equivalence class with a bounded function $g$. 
Theorem 2.10 in the textbook applies to $g$ and says that there exists a sequence of simple functions $\phi_k$ such that $\phi_k \convu g$. 
By part (c) proven above, this implies that $\v \phi_k - f\v_\infty\to 0$.

\problem{6.32}
Given that $K\in L^2(\mu\times \nu)$ we can assert that $|K|^2\in L^1(\mu\times \nu)$. 
Then Fubini-Tonelli applies to $|K|^2$ and in particular, we get that $|K_x|^2\in L^1(\nu)$ for a.e. $x\in X$.

Next, we use that $f\in L^2(\nu)$ and thus for a.e. $x\in X$: 
$$|K_x f| \leq \max\{|K_x|^2,|f|^2\}$$
And this means that $Tf(x) = \int |K(x,y) f(y)| d\nu(y)$ exists for a.e. $x\in X$.

Finally, observe that:
\begin{align}
\int \left| \int K_x f\,d\nu \right|^2 d\mu &\leq \int \left( \int |K_x f|\,d\nu \right)^2 d\mu\\
& \leq \int \v K_x\v_2^2 \v f\v_2^2 \, d\mu \label{ineq.2}
\end{align}
We get (\ref{ineq.2}) because $K_x, f\in L^2$.
At this point it is possible to flip the integral corresponding to $\v K_x\v_2 $ so we get:
 $$\int \left| \int K_x f\,d\nu \right|^2 d\mu \leq \v f\v_2^2\int | K(x,y)|^2\, d(\mu\times \nu) $$
And taking square roots of both sides:
$$\v Tf\v_2 \leq \v f\v_2\v K\v_2$$
Which is the result we want.

\problem{8.13}
\subpart{(a)}
When $\kappa =0$, we get:
$$\hat f(0) = \int_0^1 (1/2 -x) dx=0$$ 
On the other hand, if $\kappa \neq 0$ then:
\begin{align*}
\hat f(\kappa) = \int_0^1 (1/2-x)e^{-2\pi i \kappa x}dx &= \frac {-1}{(2\pi i \kappa)^2} (2\pi i \kappa -1)e^{-2\pi i \kappa x}\Big |^1_0\\
               &=\frac {-1}{2\pi i \kappa}
\end{align*}

\subpart{(b)}
As suggested in the hints document, by virtue of $\{ 2\pi i \kappa x \from \kappa\in \ZZ \}$ being an orthonormal basis of $L^2[0,1]$; then $f$ can be written like so:
\begin{align*}
f(x) &= \sum_{\kappa \in \ZZ} \inprod {e^{2\pi i \kappa x}}{f(x)} e^ {2\pi i \kappa x}\\
     &= \sum_{\kappa \neq 0}\frac {e^{-2\pi i \kappa x}}{2\pi i \kappa}
\end{align*}
First, we compute the norm using $f$:
$$\v f\v_2^2= \int_0^1 (1/2 -x)^2dx = 1/12$$
Next using Parseval's formula:
$$\left\v \sum_{\kappa \neq 0}\frac {e^{-2\pi i \kappa x}}{2\pi i \kappa}\right\v^2_2=\frac 2{4\pi^2}\sum_{\kappa=1}^\infty \frac 1{\kappa^2}$$
Equating the two expressions gives the result.
