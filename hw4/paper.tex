\noindent\textbf{Real Analysis II Assignment 4 \hspace{\fill} Luis Berlioz}
\problem{27}
First, we detail the contruction of a Generalized Cantor Set of measure 1/2. These sets have measure:
$$m(C) = \prod_k (1-\alpha_k)$$
Taking $\alpha_k= (2^k+2)^-1$, we get:
\begin{align*}
\prod_{k=1}^N \left( 1- \frac 1{2^k+2} \right) &= \frac 34 \cdots \frac{2^k+1}{2(2^{k-1} +1)}\frac{2^{k+1}+1}{2(2^{k} +1)}\frac{2^{k+2}+1}{2(2^{k+1} +1)} \cdots \frac{2^N+1}{2^N+2}\\
                                &= \frac{2^N+1}{2^{N+1}} \to \frac 12
\end{align*}
Starting with the interval [0,1]  we remove the \emph{open} middle segment of length $1/2^{2n}$. 
The resulting set $C$ is closed because its the intersection of closed sets; thus it is measurable. 
It is also nowhere-dense, since for any two points in $C$, the distance between them is greater than $1/2^{2k}$ for some $k$.
And according to the definition, an open segment of this length necesarily was removed. 

Now we argue that the gaps of $C$ can be filled in with scaled down copies of $C$. 
For convenience call $\frac 14 C$ the copy of $C$ scaled down by a factor of 1/4.
Then the biggest gap of $C$ (which has length 1/4) can be half-filled by $1/4C$.
Also, the two gaps of size 1/16 and in the general case, the $2^{n-1}$ gaps of size $2^{2n}$. 
Note that the gap's measure is $$1/2 = \sum_{k=1} \frac{2^{k-1}}{2^{2k}}$$
And filling up each one individually fills up half of the space unused by $C$.

We can repeat this process inductively recursively filling up the half of the space left at each iteration. 
This means we can repeat the process in order to be arbitrarily close to 1. 
Since we only have used scaled copies of $C$, the union of them is meager and of measure 1.
Therefore the residual is of measure zero.

\problem{29}

\problem{32}
