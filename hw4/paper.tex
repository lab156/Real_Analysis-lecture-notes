\noindent\textbf{Real Analysis II Assignment 4 \hspace{\fill} Luis Berlioz}
\problem{27}
First, we detail the construction of a Generalized Cantor Set of measure 1/2. These sets have measure:
$$m(C) = \prod_k (1-\alpha_k)$$
Taking $\alpha_k= (2^k+2)^{-1}$, we get:
\begin{align*}
\prod_{k=1}^N \left( 1- \frac 1{2^k+2} \right) &= \frac 34 \cdots \frac{2^k+1}{2(2^{k-1} +1)}\frac{2^{k+1}+1}{2(2^{k} +1)}\frac{2^{k+2}+1}{2(2^{k+1} +1)} \cdots \frac{2^N+1}{2^N+2}\\
                                &= \frac{2^N+1}{2^{N+1}} \to \frac 12
\end{align*}
Starting with the interval [0,1]  we remove the \emph{open} middle segment of length $1/2^{2n}$. 
The resulting set $C$ is closed because its the intersection of closed sets; thus it is measurable. 
It is also nowhere-dense, since for any two points in $C$, the distance between them is greater than $1/2^{2k}$ for some $k$.
And according to the definition, an open segment of this length necessarily was removed. 

Now we argue that the gaps of $C$ can be filled in with scaled down copies of $C$. 
For convenience call $\frac 14 C$ the copy of $C$ scaled down by a factor of 1/4.
Then the biggest gap of $C$ (which has length 1/4) can be half-filled by $1/4C$.
Also, the two gaps of size 1/16 and in the general case, the $2^{n-1}$ gaps of size $2^{2n}$. 
Note that the gap's measure is $$1/2 = \sum_{k=1} \frac{2^{k-1}}{2^{2k}}$$
And filling up each one individually fills up half of the space unused by $C$.

We can repeat this process inductively recursively filling up the half of the space left at each iteration. 
This means we can repeat the process in order to be arbitrarily close to 1. 
Since we only have used scaled copies of $C$, the union of them is meager and of measure 1.
Therefore the residual is of measure zero.

\problem{29}
\subsection*{Part (a)}
$\vesp$ is a subspace of $\vespy$. To see this take any two elements $x,y\in \vesp$:
$$\sum n|\alpha x(n)+ \beta y(n)| \leq \alpha\sum n|x(n)| +\beta \sum n|y(n)|<\infty$$
Also, $x\in \vesp$ implies $x\in \vespy$:
$$\v x\v = \sum |x(n)| \leq \sum n|x(n)|< \infty$$
It is a \emph{proper} subset since $f(n)=1/n^2$ is an element of $\vespy$ but not of $\vesp$: 
\begin{gather*}
    \v f\v = \sum \frac 1{n^2} = \frac {\pi^2}{6} \\
    \sum n\left| \frac 1{n^2}\right| = \sum \frac 1n = \infty
\end{gather*}
Finally, $\vesp$ is dense in $\vespy$: Take any $x\in \vespy$; then define the truncated sequence for $x$:
$$x_n = x(1),x(2),\ldots,x(n), 0,0,\ldots$$
We know $\v x\v = \sum | x(n)|<\infty$. Thus $\forall \epsilon>0$ there is an $N$ such that:
$$\sum_{N+1}^\infty |x(n)| <\epsilon$$
Therefore $\v x_N - x\v<\epsilon$.

To be complete, $\vesp$ would have to be closed in $\vespy$. Being dense and proper at the same time, this is impossible.
\subsection*{Part (b)}
To show that $T$ is unbounded, we use the following sequence:
\[x_n(k) = \begin{cases}
    \frac 1{k^2}; &\quad k\leq n\\
    0           ; &\quad k>n
\end{cases}\]
Note that $x_k \to x\in \vespy\backslash\vesp$ and, as was shown above, $\v Tx_n\v\to \infty$ while it checks that $\v x_n \v \to \frac{\pi^2}{6}$. Thus, $T$ is not bounded. 

Now, we prove that $T$ is closed. Specifically, we show that if $(x,y)$ is a limit point of $\vesp\times T\vesp$, and $(x_n,Tx_n)\to (x,y)$; then $Tx=y$. Observe that no matter the norm we choose for $\vesp\times T\vesp$ we can always say:
\begin{equation}\label{eq.1}
   \v y - Tx_n\v = \sum |y(k) - Tx_n(k)| \to 0
   \end{equation}

For any $n\in \NN$,
$$\v y - Tx\v \leq \v y - Tx_n\v + \v Tx_n - Tx\v$$
Since we know ($\ref{eq.1}$), all that is left to show is that $\v Tx_n - Tx\v\to 0$.

For any $\epsilon >0$, there exists $K\in \NN$ such that:
$$\sum_{K+1}^\infty k|x_n(k)| \leq \epsilon/3 \text{ and } \sum_{K+1}^\infty k|x(k)| \leq \epsilon/3$$
And there exists $N\in \NN$ (note $N$ depends on our choice of $K$ and $\epsilon$) such that:
$$K^2\v x_n -x\v < \epsilon/3$$
With this setup, we can conclude:
\begin{gather*}
    \v Tx_n - Tx\v = \sum_{k=1}^\infty k|x_n(k)-x(k)| \\
    \leq \sum_{k=1}^K K\v x_n-x\v + \sum_{K+1}^\infty k|x_k(k)| + \sum_{K+1}^\infty k|x(k)|  < \epsilon
\end{gather*}
\subsection*{Part (c)}
$S$ is bounded because for any $x\in \vespy$:
$$\v S x\v = \sum \frac 1k |x(k)| \leq \sum |x(k)| = \v x\v$$
And onto since for all $x\in \vesp$ we can take $y= Tx$ so that $ Sy = x$.

Lastly, we show $S$ is not open. Let's suppose for a moment it is. Then if $B_1$ is the unit ball:
$$B_r = \{f\in \vespy\from \v f \v <r\}$$
There is an $r>0$ such that $B_r \cap \vesp \subset SB_1$. Then, using that $S$ is one-to-one:
$$f\in B_1^C \implies Sf\notin SB_1 \implies Sf \notin B_r\cap \vesp$$
But the $e_n(k)=\delta_{k,n}$ have norm 1 for all $n\in \NN$. And in spite of this, there exists $e_N$ such that $\v Se_N\v = \frac 1N <r$. Therefore $S$ is not open.
\problem{32}
We need to show that there are $C,c>0$ such that for all $x\in\vesp$
$$c\v x\v_1 \leq \v x \v_2 \leq C\v x\v _1$$
By hypothesis, $c=1$.

Set $T_y\from (\vesp,\v\ \v_1) \to (\vesp,\v\ \v_2)$ defined by:
$$T_yx= \v y \v_1 x$$
Taking all $y\neq 0$ it checks that:
$$\sup_{y\neq 0} \left\v \frac{y}{\v y\v_1} x\right\v_2<\infty$$
Then by the Uniform Boundedness Principle:
$$\sup_{y\neq 0} \left\v \frac y{\v y \v_1}\right\v_2 =C< \infty$$
Therefore
$$\v y\v_2 \leq C\v y\v_1$$
