\problem{7}
\begin{enumerate}[(a)]
\item First we establish the following property:
$$T\sum_{n=0}^N (I-T)^n = \left(\sum_{n=0}^N (I-T)^n\right)T= I-(I-T)^{N+1}$$
to avoid dealing with the powers of $(I-T)$ we will temporarily rename it like so $S= I-T$. Thus
\begin{align*} 
    (I-S)\sum_{n=0}^N S^n &= I+S+S^2+S^3+S^4+\ldots + S^N\\[-1em]
            &\phantom{=I}-S-S^2-S^3-S^4-\ldots -S^N-S^{N+1}\\
            &=I-S^{N+1}
\end{align*}
Now, since $\vesp$ is complete, it means that $L(\vesp,\vesp)$ is also complete. 
We will use this to show that $\sum_{n=0}^N (I-T)^n$ converges in $L(\vesp,\vesp)$.
Indeed, note that $\v(I-T)^n\v\leq \v I-T\v^n$, this means:
$$\sum_{n=0}^\infty \v(I-T)^n\v \leq \sum_{n=0}^\infty \v I-T\v^n=\frac 1{1-\v I-T\v}<\infty$$
And this absolute convergence implies $\sum_{n=0}^\infty (I-T)^n\in L(\vesp,\vesp)$.

To show that $T$ and $\sum_{n=0}^\infty (I-T)^n$ are each other's inverse we check that their composition is the identity:
$$\left(T\sum_{n=0}^\infty (I-T)^n\right)x=\lim_{N\to \infty} \left(T\sum_{n=0}^N (I-T)^n\right)x= \lim_{N\to\infty} x-(I-T)^{N+1}x=x$$
The same is true ``mutatis mutandis'' for $\left(\sum_{n=0}^\infty (I-T)^n\right)T$.
The last equality since $ \v(I-T)^{N+1}x\v \leq \v I-T\v^{N+1}x\v\to 0$ as $N\to \infty$.

\item According to the hypothesis:
$$\v T^{-1}S-I\v \leq \v T^{-1}\v \v S-T\v <1$$
Idem if we factor on the other side $ST^{-1}$.
Then, by the preceding result $ST^{-1}$ is invertible. 

Now we prove that $S$ is onto, injective and bounded. 
To see $S$ is one to one, take any two $x,y\in\vesp$ such that:
$$Sx= Sy$$
We can multiply $T^{-1}$ on both sides,
$$T^{-1}Sx=T^{-1}Sy$$
And, since $T^{-1}S$ is invertible, $x=y$.

Showing that $S$ is surjective is simple: For any $x\in \vesp$ there is a $y\in\vesp$ such that 
$$x=T(T^{-1}Sy)=Sy$$

Finally, we check that $S$ is bounded from above:
$$\v Sx \v = \v ST^{ -1}Tx\v \leq \v ST^{ -1}\v \v T\v \v x\v$$
And bounded from below:
$$\v x \v = \v \left( T^{-1} S\right)^{-1}\left(T^{-1}S\right)x\v\leq \v(T^{-1}S)^{-1}\v \v T^{-1}\v \v Sx\v$$

\end{enumerate}

\problem{15}
\begin{enumerate}[(a)]
    \item Take any two elements $x,y\in\NNN(T)$. The set $\NNN(T)$ is closed under sum and scalar multiplication since:
        $$T(\alpha x +\beta y ) = \alpha T(x) + \beta T(y) = 0$$
        The kernel or \emph{null space} of $T$ is also closed when $T$ is continuous. 
        To see this, take any element of the closure $\bar x \in \bar{\NNN(T)}$.
        Then since $\vesp$ is connected (being a vector space); there is a sequence $\{x_n\}\in \vesp$ such that $x_n\to \bar x$ such that:
        $$\lim_{n\to \infty} T(x_n) = T(\bar x)$$
        Since we know $T(x_n)=0$ for all $x_n$ by assumption; the limit is clearly 0.
    \item First we prove that $S$ is well-defined. 
        Note that $\pi\from \vesp \to \vesp/\NNN(T)$ is a surjective linear map. 
        This means that any  equivalence class in $\NNN(T)$ can be written as $\pi(x)$ some $x\in \vesp$.
        Take any two equivalent elements $\pi(x) =\pi(x')$; then we see that $S\pi(x) - S\pi(x') = T(x)-T(x')=T(x-x')=0$ since $x-x'\in \NNN(T)$.

        $S$ is linear since $S\pi(\alpha x) = T(\alpha x)= \alpha T(x) = \alpha S\pi(x)$ and:
        $$S(\pi(x)) + S(\pi(y))= T(x) + T(y) = T(x+y) = S\pi(x+y) = S(\pi(x) + \pi(y))$$
        And $S$ is unique because it is completely determined by $T$. 
        If, for the sake of contradiction, we assume that there is $S\neq S'$, this means there exists $\pi(x)$ such that:
        $$S\pi(x) \neq S'\pi(x) \text{ and } S\pi(x) = T(x) = S'\pi(x)$$

        To prove that $\v S \v = \v T\v$, first:
        $$\v T \v = \v S\circ \pi\v \leq \v S \v $$
        Since $\v \pi \v= 1$. To check this last remark, first we note that in the \emph{unit ball} of $\vesp$ we always have $\v \pi (x)\v\leq 1$. To check the other inequality; let $\v x\v=1$ then for any $\epsilon > 0$ there exists $\pi( x') =\pi(x)$ such that:
        $$1\leq \v x' \v < \v \pi(x) \v +\epsilon$$
        This is because after adding $\epsilon$ to $\v \pi(x) \v$ it is no longer the infimum anymore; so their is room to slide $x'$ in the middle. We are done once we notice that $1-\epsilon < \v \pi(x) \v\leq 1$ which implies $\v \pi\v=1$.

        We will prove the second half, namely that $\v T \v \geq \v S\v$ using a similar argument. For all $\epsilon>0$ there is $\pi x_\epsilon\in \vesp/\NN$ such that $\v \pi x_\epsilon\v =1$ and:
        $$\v S\v -\epsilon < \v S \pi x_\epsilon\v=\v Tx_\epsilon \v\leq (1+\epsilon)\v T\v$$
        In the last inequality we used that there exists $\pi x'_\epsilon = \pi x_\epsilon $ such that $1\leq \v x'_\epsilon \v \leq 1+\epsilon$.
\end{enumerate}


\problem{17}
Assume first that $f$ is bounded, then the hypothesis of problem 15 are satisfied. That is, $\vesp$ and $\RR$ are normed vector spaces and $f\in L(\vesp, \RR)$. Then by part (a) we know that $f^{-1}(0) = \{ x\in \vesp \from f(x) =0 \}$ is a closed subspace of $\vesp$.

On the other hand, if we assume that $f^{-1}(0)$ is a closed subspace, we use the usual norm on $\vesp/f^{-1}(0)$:
$$\v x+f^{-1}(0)\v = \inf_{m\in f^{-1}(0)}\v x + m\v$$
We will show that $f$ is continuous at 0. Suppose $f$ is not the zero functional (in any, case $f=0$ is continuous). Then we have $f(x)=1$ for some $x\in\vesp$. Also, for every $\epsilon >0$ there exists $x'$ such that $x'=x+m$ with $m\in f^{-1}(0)$ and $\v x' \v > 1 - \epsilon$. Then take $\delta = \epsilon$.
