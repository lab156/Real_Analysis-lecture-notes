\problem{7}
\begin{enumerate}[(a)]
\item First we establish the following property:
$$T\sum_{n=0}^N (I-T)^n = \left(\sum_{n=0}^N (I-T)^n\right)= I-(I-T)^{N+1}$$
to avoid dealing with the powers of $(I-T)$ we will temporarly rename it like so $S= I-T$. Thus
\begin{align*} 
(I-S)\sum_{n=0}^N S^n &= I+S+S^2+S^3+S^4+\ldots + S^N\\
            &=\phantom{I}-S-S^2-S^3-S^4-\ldots -S^N-S^{N+1}\\
            &=I-S^{N+1}
\end{align*}
Now, since $\vesp$ is complete, it means that $L(\vesp,\vesp)$ is also complete. 
We will use this to show that $\sum_{n=0}^N (I-T)^n$ converges in $L(\vesp,\vesp)$.
Indeed note that $\v(I-T)^n\v\leq \v I-T\v^n$, this means:
$$\sum_{n=0}^\infty \v(I-T)^n\v \leq \sum_{n=0}^\infty \v I-T\v^n=\frac 1{1-\v I-T\v}<\infty$$
And this absolute convergence implies $\sum_{n=0}^\infty (I-T)^n\in L(\vesp,\vesp)$.

To show that $T$ and $\sum_{n=0}^\infty (I-T)^n$ are each other's inverse we check that their composition is the identity:
$$\left(T\sum_{n=0}^\infty (I-T)^n\right)x=\lim_{N\to \infty} \left(T\sum_{n=0}^N (I-T)^n\right)x= \lim_{N\to\infty} x-(I-T)^{N+1}x=x$$
The same is true ``mutatis mutandis'' for $\left(\sum_{n=0}^\infty (I-T)^n\right)T$.
The last equality since $ \v(I-T)^{N+1}x\v leq \v I-T\v^{N+1}x\v\to 0$ as $N\to \infty$.

\item According to the hypothesis:
$$\v T^{-1}S-I\v \leq \v T^{-1}\v \v S-T\v <1$$
Idem for $ST^{-1}$.
Then, by the preceding result $ST^{-1}$ is invertible. 

Now we prove that $S$ is onto, injective and bounded. 
To see $S$ is one to one, take any two $x,y\in\vesp$ such that:
$$Sx= Sy$$
We can multiply $T^{-1}$ on both sides,
$$T^{-1}Sx=T^{-1}Sy$$
And since $T^{-1}S$ is invertible, $x=y$.

Showing that $S$ is surjective is simple: For any $x\in \vesp$ there is a $y\in\vesp$ such that 
$$x=T(T^{-1}Sy)=Sy$$

Finally, it checks that $S^{-1}$ is a linear map, to check it is bounded, 
\end{enumerate}
