\noindent\textbf{Real Analysis II Assignment 9 \hspace{\fill} Luis Berlioz}
\problem{3}
\begin{enumerate}[(i)]
\item To check that such a sequence exists, first note that for all $n\in \NN$ there exists $x_n\in \HH$ such that:
$$\Big| |\inprod{Tx_n}{x_n}| - \v T\v\Big|<\frac 1n$$
If this was not the case, then there would be a smaller upper bound for $\{ \inprod {Tx}x\from \v x\v=1\}$ contradicting the step 1 of the handout.
Next, since the closure of $TS=\{Tx\from \v x\v =1\}$ is compact (in the metric topology), we can assert that $(Tx_n)$ has a limit point $y$.
Thus, there is a subsequence of $(Tx_n)$ that converges strongly to $y$.
 
\item  According to the Banach-Alaoglu Theorem, and because $\HH=\HH^*=\HH^{**}$; we know  that the closed unit ball $B$ is weakly compact (this is a good place to require the separability of $\HH$).
    Thus, since all $x_n$ have norm 1, there is a subsequence that converges to say $a\in \HH$. 
    To show that $\v a\v\leq 1$, notice that 
    $$\inprod{x_n}{\frac a{\v a\v}}\leq \v x_n\v\, \left\v \frac a{\v a\v}\right\v \leq 1$$
    and as $n\to \infty$:
    $$\inprod{x_n}{\frac a{\v a\v}}\to \inprod{a}{\frac a{\v a\v}}=\v a\v\leq 1$$
\item We know that $x_n\xrightarrow{weak} a$ means that $Tx_n\xrightarrow{weak} Ta$ by hint (ii) of problem 1 in the handout.
    Finally note that:
    \begin{gather*}
        |\inprod{Tx_n}{x_n} - \inprod{Ta}a| \\
        \leq |\inprod{Tx_n}{x_n} - \inprod{Ta}{x_n}|+ |\inprod{Ta}{x_n} - \inprod{Ta}a|
    \end{gather*}
    The weak convergence of both $(x_n)$ and the $(Tx_n)$ imply that as a sequence of real numbers $|\inprod{Tx_n}{x_n}| \to  |\inprod{Ta}a|$.
\end{enumerate}

\problem{4}
Let $\HH=\CC^2$ and let $T$ be the operator of rotation by 90$^\circ$.
$$T=\begin{bmatrix}
    0 & -1 \\
    1 & 0 
\end{bmatrix}$$
If $x=[a+bi,c+di]\in \HH$ then:
$$\inprod {Tx}x = (-c-di)(a-bi) + (a+bi)(c-di) =0$$

\problem{5}
First we show that $T$ is Hermitian. Let $f,g\in \HH$ then, by direct computation:
$$\inprod {Tf}g = \int_0^1 (tf)g = \int_0^1 f(tg) = \inprod {f}{Tg}$$
Next, observe that for $0\leq t\leq 1$, we have that $t^2f^2(t)\leq f^2(t)$. Thus:
$$\v Tf\v^2 = \int_0^1 t^2f^2(t) \leq \int_0^1 f^2(t) = \v f\v$$
This implies $\v T\v\leq 1$. Moreover, let $f_n\in \HH$ defined as:
$$f_n(t) = \sqrt{n} \chi_{[1-1/n,1]}$$
is shows that $\v T\v$ is exactly 1 because:
\begin{align*}
    \v T f_n\v &= \int_0^1 t^2 f_n^2 = \frac 13 t^3 (n-(1-1/n)^3)\\
               &= 1- \frac 1{n} + \frac 1{n^2}
\end{align*}
That gets arbitrarily close to 1.

Finally, $T$ does not reach its norm. If, for the sake of contradiction, we assume it does at the eigenvector $f\in \HH$ then the equation $t\,f(t) = \lambda f(t)$ implies that $f=0$ a.e. This is impossible because eigenvectors are nonzero.
