\noindent\textbf{Real Analysis II Assignment 6 \hspace{\fill} Luis Berlioz}
\problem{45}
First,  define a seminorm on $C^\infty(\RR)$:
\begin{equation}
    p_{j,k}(f)= \sup_{x\in [-j,j]} \left\{ |f^{(k)}(x) |\right\} \label{p.def}
\end{equation}
We need check that $p_{j,k}$ is defined for every $f\in C^\infty(\RR)$.
Indeed, this is the case because the supremum of a continuous function ($f^{(k)}$) is taken in a compact set ($[-j,j]$).
We conclude that the supremum in (\ref{p.def}) always exists.

Now we can check that the functions $p_{j,k}(f)$ are in fact seminorms. 
Non-negativity, and Homogeneity, is guaranteed because the supremum is taken from the absolute value of the $k-$th derivative of the function.
For the triangle inequality, we check that:
\begin{align*}
p_{j,k}(f+g) & = \sup_{x\in [-m,m]} \left\{ |f^{(k)}(x) + g^{(k)}(x)  | \right\}\\
&\leq \sup_{x\in [-m,m]} \left\{ |f^{(k)}(x) | +|g^{(k)}(x) |  \right\}\\
&\leq p_{j,k}(f) + p_{j,k}(g)
\end{align*}

Next, we show that the family of seminorms defined before is compatible with the given definition of convergence. 
Note that as a family of seminorms, $\{p_{j,k} \from j,k\in \NN\}$ defines a topology in which $f_n\to f$ iff (Theorem 5.14 part (b)):
$$\forall j,k\in\NN\quad p_{j,k}(f_n-f)\to 0$$
And, since for a fixed $k$ this is just the topology of uniform convergence in all compact set, we can conclude that 
$$\forall k\in \NN ,\ f_n^{(k)}\xrightarrow{unif.}  f^{(k)}$$

Lastly, we prove that $C^\infty(\RR)$ is complete under this topology.
Let $(f_n)_{n=1}^{\infty}$ a be Cauchy sequence in the sense of the seminorms $p_{j,k}$.
Fixing any $k$ implies that the $k-$derivatives converge uniformly to some function we call $g_k$.
We will prove that, in fact, they are all derivatives of the the same function. We will prove it  by induction on the order of the derivative ($k$).
To see the base case, notice that $f_n\to g_0$ and $f_n' \xrightarrow{unif.} g_1$; thus, $g_0'=g_1$. 
Next, if we assume that: 
$$\frac{d^k g_0}{dx^k}=g_{k}$$ 
Then again, using that  $f^{(k+1)}\xrightarrow{unif.}g_{k+1}$ we get that $g_k'=g_{k+1} $.
Therefore the sequence $(f_n)_{n=1}^{\infty}$ converges in the sense of the seminorms $p_{j,k}$ to a function $g_0 \in C^\infty(\RR)$.

\problem{47}
\subsection*{Part (b)}
Let $\{x_n \from n\in \NN\}\subset \vesp$ and $x_n \xrightarrow{weak} x\in \vesp$. This is the same as saying that for all $f\in \vesp^*$, we have the ``normal convergence'' of a sequence of real numbers:
$$f(x_n) \to f(x) \text{ as } n\to \infty$$  
If we view the $x_n$ as linear functionals $x_n\from \vesp^*\to \RR$ then fixing any $f\in \vesp^*$ we get that the following two sets are equal:
$$\{x_n(f)\from n\in \NN\} = \{f(x_n)\from n\in \NN\}$$
As noted above, this set is bounded since the $f(x_n)$ are a sequence of real numbers that converges to $f(x)$. Having established the above we can use the Uniform Boundedness Principle\footnote{I got this idea from the Wikipedia article on Weak Convergence} since $\vesp^*$ is a Banach space, we get that:
$$\sup_{n\in \NN}\v x_n(f)\v < \infty \implies \sup_{n\in \NN}\v x_n \v < \infty $$

Similarly for $\{f_n \from n\in \NN\}\subset \vesp^*$ and $f_n\xrightarrow{weak^*} f$, and using that $\vesp$ is assumed to be Banach:
$$\sup_{n\in \NN}\v f_n(x)\v < \infty \implies \sup_{n\in \NN}\v f_n \v < \infty $$
Therefore, $\{f_n\}$ is bounded.

\problem{48}
\subsection*{Part (a)}
Let $x\in \bar B$ where the closure is taken in the sense of the weak topology. Then, there must be a sequence in $B$ that converges weakly to $x$, namely:  
$$B \ni x_n\xrightarrow{weak^*} x$$
Take $f\in \vesp^*$ such that $f(x)=\v x\v$ (that we know exists because of Theorem 5.8 (b)) then: $\lim_n f(x_n)\to f(x) = \v x\v$. If we also consider that, by assumption, for all $n\in \NN$:
$$f(x_n) \leq \v x_n \v \leq 1$$
Finally, $\forall \epsilon>0\ \exists N$ such that for $n\geq N$, $|f(x_n) - \v x\v|<\epsilon$ then:
$$\v x\v < \epsilon +f(x_n)\leq \epsilon +1$$
Which implies that  $\v x\v\leq 1$; therefore, $B$ is closed in the weak topology.

\problem{55}
\subsection*{Part (a)}
Notice that, as given, the identity is \emph{false}; for taking $x\neq y =0$ gives: $\inprod xy = 0$ on the other hand:
$$\v x + 0\v^2 +\v x - 0\v^2 + i\v x +i0\v^2 - i\v x-i0\v^2= 2\v x\v^2$$
The identity is known to be true in the case the first addition is replaced by a minus sign. In this case, the proof requires only basic computation.
